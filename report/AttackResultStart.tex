\documentclass{article}
\usepackage{graphicx}
\usepackage[margin=0.85in]{geometry}
\usepackage[english]{babel}
\usepackage{float}
\usepackage{amsmath}
\usepackage{url}
\usepackage{cite}
\usepackage{amssymb}
\usepackage{booktabs}
\usepackage{tabularx}
\usepackage{hyperref}

\title{Trust Model Attack Results}
\author{Cody Lewis}
\date{\today}

\begin{document}
  \maketitle
  \tableofcontents
  \newpage
  This report shows the results of simulations of the trust model specified
  in\cite{saied13}. These results show the impact of varying percentages of
  malicious nodes on the system, they also the show effects of context based
  attacks.
  \section{Parameters of all the simulations}
    \begin{tabularx}{\textwidth}{X X}
        \toprule
        $ \theta{} $ & $ 0.7 $ \\
        \midrule
        $ \lambda{} $ & $ 0.7 $ \\
        \midrule
        $ \eta{} $ & $ 1 $ \\
        \midrule
        Total nodes & $ 200 $ \\
        \midrule
        Transactions & $ 300 $ \\
        \midrule
        Poor Witnesses & $ 20\% $ \\
        \midrule
        Constrained Nodes & $ 50\% $ \\
        \bottomrule
    \end{tabularx} \\

    \section{Differing Reputations}
    The simulations where performed with varying reputation thresholds,
    with thresholds other than -1 the simulations tended to have to simulated
    network become unable to perform anymore transactions as all nodes had
    fallen below thr threshold. These points are made note of in the
    reputation plots of the simulations effected.\


