

    \section{Reputation Attacks}
        Reputation can also be attacked using context, if a client node chooses
        a target context that is a sufficient distance from the selected
        proxy's abilities, then the client can create a faulty report without
        the repercussions to QR as the report's distance causes it to be
        restricted from the QR calculation. The client is not completely immune
        to repercussions though, as if a non-malicious node requests the same
        proxy in the same context before sufficient decay, then the descrepency
        formed may negatively effect the clients QR\@. \\
        The following plots show the changes in reputation when a single node
        is requested for transactions constantly. The node has service level 50
        and capability of 50.
        \begin{figure}[H]
            \centering
            \includegraphics[width=\columnwidth]{graphs/matching_context.png}
            \caption{Changes in context when service target and capability
            target always match what is provided}
        \end{figure}
        \begin{figure}[H]
            \centering
            \includegraphics[width=\columnwidth]{graphs/high_service_request.png}
            \caption{Changes in context when service target is always higher
            than what is provided, but capability target always matchs}
        \end{figure}
        \begin{figure}[H]
            \centering
            \includegraphics[width=\columnwidth]{graphs/high_service_bad_mouth_request.png}
            \caption{Changes in context when service target and capability
            target always match what is provided, and the proxy server is
            always bad mouthed}
        \end{figure}


    \begin{thebibliography}{9}
        \bibitem{saied13}
            Ben Saied Y,
            \textit{Trust management system design for the Internet of Things: A context aware and multi-service approach},
            Computers \& Security (2013),
            \url{http://dx.doi.org/10.1016/j.cose.2013.09.001}
    \end{thebibliography}
\end{document}
